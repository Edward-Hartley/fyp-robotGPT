%%%%%%%%%%%%%%%%%%%%%%%%%%%%%%%%%%%%%%%%%
% University Assignment Title Page 
% LaTeX Template
% Version 1.0 (27/12/12)
%
% This template has been downloaded from:
% http://www.LaTeXTemplates.com
%
% Original author:
% WikiBooks (http://en.wikibooks.org/wiki/LaTeX/Title_Creation)
%
% License:
% CC BY-NC-SA 3.0 (http://creativecommons.org/licenses/by-nc-sa/3.0/)
% 
%
%%%%%%%%%%%%%%%%%%%%%%%%%%%%%%%%%%%%%%%%%
%----------------------------------------------------------------------------------------
%	PACKAGES AND OTHER DOCUMENT CONFIGURATIONS
%----------------------------------------------------------------------------------------
\usepackage[usenames,dvipsnames]{xcolor}
\usepackage[a4paper,hmargin=2.0cm,vmargin=2.0cm,includeheadfoot]{geometry}
\usepackage{textpos}
\usepackage[style=numeric, backend=biber]{biblatex} % for bibliography
\usepackage{csquotes}
\usepackage{tabularx,longtable,multirow,subfigure,caption}%hangcaption
\usepackage{fancyhdr} % page layout
\usepackage{url} % URLs
\usepackage[english]{babel}
\usepackage{amsmath}
\usepackage{graphicx}
\usepackage{svg}
\usepackage{tcolorbox}
\usepackage{dsfont}
\usepackage{epstopdf} % automatically replace .eps with .pdf in graphics
\usepackage{array}
\usepackage{latexsym}
\usepackage[pdftex,hypertexnames=false,colorlinks]{hyperref} % provide links in pdf
\usepackage{opensans}
\usepackage[pdf]{graphviz}
\usepackage{float}
\usepackage{listings}

\addbibresource{references.bib}

\hypersetup{pdftitle={},
  pdfsubject={}, 
  pdfauthor={},
  pdfkeywords={}, 
  pdfstartview=FitH,
  pdfpagemode={UseOutlines},% None, FullScreen, UseOutlines
  bookmarksnumbered=true, bookmarksopen=true, colorlinks,
    citecolor=black,%
    filecolor=black,%
    linkcolor=black,%
    urlcolor=black}

\usepackage[all]{hypcap}


%\usepackage{color}
%\usepackage[tight,ugly]{units}

%\usepackage{tcolorbox}
%\usepackage[colorinlistoftodos]{todonotes}
% \usepackage{ntheorem}
% \theoremstyle{break}
% \newtheorem{lemma}{Lemma}
% \newtheorem{theorem}{Theorem}
% \newtheorem{remark}{Remark}
% \newtheorem{definition}{Definition}
% \newtheorem{proof}{Proof}


%%% Default fonts
\renewcommand*{\rmdefault}{bch}
\renewcommand*{\ttdefault}{cmtt}



%%% Default settings (page layout)
\setlength{\parindent}{0em}  % indentation of paragraph

\setlength{\parindent}{0em}  % indentation of paragraph

\setlength{\headheight}{14.5pt}
\pagestyle{fancy}
\renewcommand{\chaptermark}[1]{\markboth{\chaptername\ \thechapter.\ #1}{}} 
%\fancyhead[RO]{\sffamily \textbf{\thepage}} %Page no.in the right on even pages
%\fancyhead[LE]{\sffamily \textbf{\thepage}} %Page no. in the left on odd pages

\fancyfoot[ER,OL]{\thepage}%Page no. in the left on
                                %odd pages and on right on even pages
\fancyfoot[OC,EC]{\sffamily }
\renewcommand{\headrulewidth}{0.1pt}
\renewcommand{\footrulewidth}{0.1pt}
\captionsetup{margin=10pt,font=small,labelfont=bf}


%--- chapter heading

\def\@makechapterhead#1{%
  \vspace*{10\p@}%
  {\parindent \z@ \raggedright \sffamily
    \interlinepenalty\@M
    \Huge\bfseries \thechapter \space\space #1\par\nobreak
    \vskip 30\p@
  }}

%--- chapter heading

\def\@makechapterhead#1{%
  \vspace*{10\p@}%
  {\parindent \z@ \raggedright \sffamily
        %{\Large \MakeUppercase{\@chapapp} \space \thechapter}
        %\\
        %\hrulefill
        %\par\nobreak
        %\vskip 10\p@
    \interlinepenalty\@M
    \Huge\bfseries \thechapter \space\space #1\par\nobreak
    \vskip 30\p@
  }}

%---chapter heading for \chapter*  
\def\@makeschapterhead#1{%
  \vspace*{10\p@}%
  {\parindent \z@ \raggedright
    \sffamily
    \interlinepenalty\@M
    \Huge \bfseries  #1\par\nobreak
    \vskip 30\p@
  }}	
\allowdisplaybreaks